\chapter{Basic concepts}
\label{Chp:Concepts}

The GraphBLAS C API is used to construct  
graph algorithms expressed ``in the language of linear algebra.''
Graphs are expressed as matrices, and the operations over 
these matrices are generalized through the use of a
semiring algebraic structure.

In this chapter, we will define the basic concepts used to
define the GraphBLAS C API.  We provide the following elements:

\begin{itemize}
\item Glossary of terms and notation used in this document.  
\item Algebraic structures and associated arithmetic foundations of the API.
\item Functions that appear in the GraphBLAS algebraic structures and how they are managed.
\item Domains of elements in the GraphBLAS.
\item Indices, index arrays, scalar arrays, and external matrix formats used to expose the contents of GraphBLAS objects.
\item The GraphBLAS opaque objects. 
\item The execution and error models implied by the GraphBLAS C specification.
\item Enumerations used by the API and their values.
\end{itemize}

\section{Glossary}

%=============================================================================

\subsection{GraphBLAS API basic definitions}

\glossBegin

\glossItem{operator} A function that performs an operation on the scalar elements 
stored in GraphBLAS matrices, vectors, and scalars.

\glossItem{GraphBLAS operation} A mathematical operation defined in the
GraphBLAS mathematical specification. These operations (not to be confused with \emph{operators}) typically act
on matrices and vectors with elements defined in terms of an algebraic semiring. 
\glossEnd

%=============================================================================

\subsection{GraphBLAS objects and their structure}

\glossBegin
\glossItem{domain} The set of valid values for the elements stored in a 
GraphBLAS \emph{collection} or operated on by a GraphBLAS \emph{operator}.
Note that some GraphBLAS objects involve functions that map values from 
one or more input domains onto values in an output domain.  These GraphBLAS 
objects would have multiple domains.

\glossItem{collection} An opaque GraphBLAS object that holds a number of
elements from a specified \emph{domain}. Because these objects are based on an 
opaque datatype, an implementation of the GraphBLAS C API has the flexibility 
to optimize the data structures for a particular platform.  GraphBLAS objects 
are often implemented as sparse data structures, meaning only the subset of the
elements that have values are stored.

\glossItem{implied zero}  Any element that has a valid index (or indices) 
in a GraphBLAS vector or matrix but is not explicitly identified in the list of 
elements of that vector or matrix. From a mathematical perspective, an
\emph{implied zero} is treated as having the 
value of the zero element of the relevant monoid or semiring.
However, GraphBLAS operations are purposefully defined using set notation in such a way
that it makes it unnecessary to reason about implied zeros. 
Therefore, this concept is not used in the definition of GraphBLAS methods and operators.

\glossItem{mask} An internal GraphBLAS object used to control how values 
are stored in a method's output object.  The mask exists only inside a method; hence,
it is called an \emph{internal opaque object}.  A mask is formed from the elements of
a collection object (vector or matrix) input as a mask parameter to a method. GraphBLAS 
allows two types of masks:
\begin{enumerate}
\item In the default 
case, an element of the mask exists for each element that exists in the 
input collection object when the value of that element, when cast to a Boolean type, evaluates to 
{\tt true}.  
\item In the {\it structure only} case, masks have structure but no values. 
The input collection describes a structure whereby an 
element of the mask exists for each element stored in the input collection regardless of its value.
\end{enumerate}

\glossItem{complement} The \emph{complement} of a 
GraphBLAS mask, $M$, is another mask, $M'$, where the elements of $M'$
are those elements from $M$ that \emph{do not} exist.  
\glossEnd

%=============================================================================

\subsection{Algebraic structures used in the GraphBLAS}

\glossBegin
\glossItem{associative operator} In an expression where a binary operator is used 
two or more times consecutively, that operator is \emph{associative} if the result 
does not change regardless of the way operations are grouped (without changing their order). 
In other words, in a sequence of binary operations using the same associative 
operator, the legal placement of parenthesis does not change the value resulting 
from the sequence operations.  Operators that are associative over infinitely 
precise numbers (e.g., real numbers) are not strictly associative when applied to 
numbers with finite precision (e.g., floating point numbers). Such non-associativity 
results, for example, from roundoff errors or from the fact some numbers can not 
be represented exactly as floating point numbers.   In the GraphBLAS specification, 
as is common practice in computing, we refer to operators as \emph{associative} 
when their mathematical definition over infinitely precise numbers is associative 
even when they are only approximately associative when applied to finite precision 
numbers.

No GraphBLAS method will imply a predefined grouping over any associative operators. 
Implementations of the GraphBLAS are encouraged to exploit associativity to optimize 
performance of any GraphBLAS method with this requirement. This holds even if the 
definition of the GraphBLAS method implies a fixed order for the associative operations.

\glossItem{commutative operator} In an expression where a binary operator is used (usually
two or more times consecutively), that operator is \emph{commutative} if the result does 
not change regardless of the order the inputs are operated on.

No GraphBLAS method will imply a predefined ordering over any commutative operators. 
Implementations of the GraphBLAS are encouraged to exploit commutativity to optimize 
performance of any GraphBLAS method with this requirement. This holds even if the 
definition of the GraphBLAS method implies a fixed order for the commutative operations.

\glossItem{GraphBLAS operators} Binary or unary operators that act on elements of GraphBLAS 
objects.  \emph{GraphBLAS operators} are used to express algebraic structures used in the 
GraphBLAS such as monoids and semirings. They are also used as arguments to several
GraphBLAS methods. There are two types of \emph{GraphBLAS operators}: 
(1) predefined operators found in Table~\ref{Tab:PredefOperators} and (2) user-defined 
operators created using {\sf GrB\_UnaryOp\_new()} or {\sf GrB\_BinaryOp\_new()} 
(see Section~\ref{Sec:AlgebraMethods}).

\glossItem{monoid} An algebraic structure consisting of one domain, an associative 
binary operator, and the identity of that operator.  There are two types 
of GraphBLAS monoids: (1) predefined monoids found in 
Table~\ref{Tab:PredefinedMonoids} and (2) user-defined monoids created using 
{\sf GrB\_Monoid\_new()} (see Section~\ref{Sec:AlgebraMethods}). 

\glossItem{semiring} An algebraic structure consisting of a set of allowed values
(the \emph{domain}), a commutative and associative binary operator called addition, a binary operator 
called multiplication (where multiplication distributes over addition),
and identities over addition (\emph{0}) and multiplication (\emph{1}).  The additive
identity is an annihilator over multiplication.   

\glossItem{GraphBLAS semiring} is allowed to diverge from the mathematically 
rigorous definition of a \emph{semiring} since certain combinations of domains, operators, and identity 
elements are useful in graph algorithms even when they do not strictly match the mathematical
definition of a semiring.
There are two types 
of \emph{GraphBLAS semirings}: (1) predefined semirings found in 
Tables~\ref{Tab:PredefinedTrueSemirings} and~\ref{Tab:PredefinedUsefulSemirings}, and (2) user-defined semirings created using 
{\sf GrB\_Semiring\_new()} (see Section~\ref{Sec:AlgebraMethods}).

\glossItem{index unary operator} A variation of the unary operator that operates
on elements of GraphBLAS vectors and matrices along with the index values 
representing their location in the objects.  There are predefined index unary
operators found in Table~\ref{Tab:PredefIndexOperators}), and user-defined
operators created using {\sf GrB\_IndexUnaryOp\_new} (see Section~\ref{Sec:AlgebraMethods}).
\glossEnd

%=============================================================================

%=============================================================================

\subsection{GraphBLAS methods: behaviors and error conditions}
\glossBegin
\glossItem{undefined behavior} Behavior that is not specified by the GraphBLAS C API.
A conforming implementation is free to choose results delivered from a method
whose behavior is undefined. 

\glossItem{dimension compatible} GraphBLAS objects (matrices and vectors) that are
passed as parameters to a GraphBLAS method are dimension (or shape) compatible if
they have the correct number of dimensions and sizes for each dimension to satisfy 
the rules of the mathematical definition of the operation associated with the method. 
If any \emph{dimension compatibility} rule above is violated, execution of the GraphBLAS 
method ends and the {\sf GrB\_DIMENSION\_MISMATCH} error is returned.

\glossItem{domain compatible} Two domains for which values from one domain can be 
cast to values in the other domain as per the rules of the C language. In particular, 
domains from Table~\ref{Tab:PredefinedTypes} 
are all compatible with each other, and a domain from a user-defined type is only 
compatible with itself. If any \emph{domain compatibility} rule above is 
violated, execution of the GraphBLAS method ends and the {\sf GrB\_DOMAIN\_MISMATCH} 
error is returned.
\glossEnd

\vfill

\newgeometry{left=2.5cm,top=2cm,bottom=2cm}

%=============================================================================
%=============================================================================

\section{Notation}

\begin{tabular}[H]{l|p{5in}}
Notation & Description \\
\hline
$\Dout, \Dinn, \Din1, \Din2$  & Refers to output and input domains of various GraphBLAS operators. \\
$\bDout(*), \bDinn(*),$ & Evaluates to output and input domains of GraphBLAS operators (usually \\
~~~~$\bDin1(*), \bDin2(*)$ & a unary or binary operator, or semiring). \\
$\mathbf{D}(*)$   & Evaluates to the (only) domain of a GraphBLAS object (usually a monoid, vector, or matrix). \\ 
$f$             & An arbitrary unary function, usually a component of a unary operator. \\
$\mathbf{f}(F_u)$ & Evaluates to the unary function contained in the unary operator given as the argument. \\
$\odot$         & An arbitrary binary function, usually a component of a binary operator. \\
$\mathbf{\bigodot}(*)$ & Evaluates to the binary function contained in the binary operator or monoid given as the argument. \\
$\otimes$       & Multiplicative binary operator of a semiring. \\
$\oplus$        & Additive binary operator of a semiring. \\
$\mathbf{\bigotimes}(S)$ & Evaluates to the multiplicative binary operator of the semiring given as the argument. \\
$\mathbf{\bigoplus}(S)$ & Evaluates to the additive binary operator of the semiring given as the argument. \\
$\mathbf{0}(*)$   & The identity of a monoid, or the additive identity of a GraphBLAS semiring. \\
$\mathbf{L}(*)$   & The contents (all stored values) of the vector or matrix GraphBLAS objects.  For a vector, it is the set of (index, value) pairs, and for a matrix it is the set of (row, col, value) triples. \\
$\mathbf{v}(i)$ or $v_i$   & The $i^{th}$ element of the vector $\vector{v}$.\\
$\mathbf{size}(\vector{v})$ & The size of the vector $\vector{v}$.\\
$\mathbf{ind}(\vector{v})$ & The set of indices corresponding to the stored values of the vector $\vector{v}$.\\
$\mathbf{nrows}(\vector{A})$ & The number of rows in the $\matrix{A}$.\\
$\mathbf{ncols}(\vector{A})$ & The number of columns in the $\matrix{A}$.\\
$\mathbf{indrow}(\vector{A})$ & The set of row indices corresponding to rows in $\matrix{A}$ that have stored values.  \\
$\mathbf{indcol}(\vector{A})$ & The set of column indices corresponding to columns in $\matrix{A}$ that have stored values. \\
$\mathbf{ind}(\vector{A})$ & The set of $(i,j)$ indices corresponding to the stored values of the matrix. \\
$\mathbf{A}(i,j)$ or $A_{ij}$ & The element of $\matrix{A}$ with row index $i$ and column index $j$.\\
$\matrix{A}(:,j)$ & The $j^{th}$ column of matrix $\matrix{A}$.\\
$\matrix{A}(i,:)$ & The $i^{th}$ row of matrix $\matrix{A}$.\\
$\matrix{A}^T$ &The transpose of matrix $\matrix{A}$. \\
$\neg\matrix{M}$ & The complement of $\matrix{M}$.\\
s$(\matrix{M})$ & The structure of $\matrix{M}$.\\
$\neg$s$(\matrix{M})$ & The complement of the structure of $\matrix{M}$.\\
$\vector{\widetilde{t}}$ & A temporary object created  by the GraphBLAS implementation. \\
\end{tabular}

\restoregeometry


\section{Mathematical foundations}

Graphs can be represented in terms of matrices. The values stored in these 
matrices correspond to attributes (often weights) of edges in the graph.\footnote{More information on the mathematical foundations can be found in the following paper: J. Kepner, P. Aaltonen, D. Bader,  A. Buluç, F. Franchetti, J. Gilbert, D. Hutchison, M. Kumar, A. Lumsdaine, H. Meyerhenke, S. McMillan, J. Moreira, J. Owens, C. Yang, M. Zalewski, and T. Mattson. 2016, September. Mathematical foundations of the GraphBLAS. In \emph{2016 IEEE High Performance Extreme Computing Conference (HPEC)} (pp. 1-9). IEEE.} 
Likewise, information about vertices in a graph are stored in vectors.
The set of valid values that can be stored in either matrices or vectors
is referred to as their domain. Matrices are usually sparse because the 
lack of an edge between two vertices means that nothing is stored at the 
corresponding location in the matrix.  Vectors may be sparse or dense, or they may 
start out sparse and become dense as algorithms traverse the graphs.

Operations defined by the GraphBLAS C API specification operate on these 
matrices and vectors to carry out graph algorithms.  These GraphBLAS 
operations are defined in terms of GraphBLAS semiring algebraic 
structures. Modifying the underlying semiring changes the result of 
an operation to support a wide range of graph algorithms.
Inside a given algorithm, it is often beneficial to change the GraphBLAS 
semiring that applies to an operation on a matrix.  This has two 
implications for the C binding of the GraphBLAS API.  

First, it means that we define a separate object for the semiring 
to pass into methods.  Since in many cases the full
semiring is not required, we also support passing monoids or
even binary operators, which means the semiring is implied rather than 
explicitly stated.

Second, the ability to change semirings impacts the meaning of 
the \emph{implied zero} in a sparse representation of a matrix or vector.
This element in real arithmetic is zero, which is the 
identity of the \emph{addition} operator and the annihilator of the
\emph{multiplication} operator.  As the semiring changes, this 
implied zero changes to the identity of the \emph{addition} operator 
and the annihilator (if present) of the \emph{multiplication} operator 
for the new semiring. Nothing changes regarding what is stored in the sparse 
matrix or vector, but the implied zeros within them change with respect to a 
particular operation. In all cases, the nature of the implied zero does not 
matter since the GraphBLAS C API requires that implementations treat them as 
nonexistent elements of the matrix or vector.

As with matrices and vectors, GraphBLAS semirings have domains
associated with their inputs and outputs.  The semirings in the 
GraphBLAS C API are defined with two domains associated with the input operands and one 
domain associated with output.  When used in the GraphBLAS C API these
domains may not match the domains of the matrices and vectors supplied in
the operations.  In this case, only valid \emph{domain compatible} casting 
is supported by the API.

The mathematical formalism for graph operations in the language of 
linear algebra often assumes that we can operate in the field of real numbers. 
However, the GraphBLAS C binding is designed for implementation on computers, 
which by necessity have a finite number of bits to represent numbers. 
Therefore, we require a conforming implementation to use floating point 
numbers such as those defined by the IEEE-754 standard (both single- and double-precision) 
wherever real numbers need to be represented. The practical implications of 
these finite precision numbers is that the result of a sequence of 
computations may vary from one execution to the next as the grouping of operands
(because of associativity) within the operations changes.  While techniques are known to 
reduce these effects, we do not require or even expect an implementation 
to use them as they may add considerable overhead. In most 
cases, these roundoff errors are not significant. When they are significant, 
the problem itself is ill-conditioned and needs to be reformulated.


