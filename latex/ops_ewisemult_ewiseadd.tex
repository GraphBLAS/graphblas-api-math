%-----------------------------------------------------------------------------
\section{Element-wise multiplication or intersection}

{\sf eWiseMult} returns an object whose indices are the ``intersection'' of 
the indices of the inputs. The operator is invoked when scalars from two 
operands are present.

%-----------------------------------------------------------------------------

\subsection{{\sf eWiseMult/Intersect}: Vector variant}

Perform element-wise (general) multiplication on the intersection of elements 
of two vectors, producing a third vector as result.

\paragraph{\syntax}

$$
\matrix{T} ~ = ~ \vector{u} \otimes \vector{v}
$$

where

\begin{itemize}[leftmargin=1.1in]
    \item[$\vector{u}$]    ({\sf IN}) $\in \mathbb{D}_{u}^{n}$

    \item[$\vector{v}$]    ({\sf IN}) $\in \mathbb{D}_{v}^{n}$

    \item[$\otimes$]   ({\sf IN}) binary operator to combine values, $\mathbb{D}_{in1} \times \mathbb{D}_{in2} \rightarrow \mathbb{D}_\otimes$.  It does not need to commutative or associative.

    \item[$\matrix{T}$]    ({\sf OUT}) $\in \mathbb{D}_\otimes^{n}$

\end{itemize}

\paragraph{Description}

This variant of {\sf eWiseMult} or {\sf Intersect} computes the element-wise ``product'' or 
``intersection'' of two vectors given a specified binary operator, 
$\otimes$: $\matrix{T} = \vector{u} \otimes \vector{v}$.  Assuming the domains of the
elements in the vectors and the domains expected by the operator are compatible
and the sizes of the vectors are the same.

From the argument vectors, the internal vectors used in 
the computation are formed ($\leftarrow$ denotes copy):
\begin{enumerate}
	\item Vector $\vector{\widetilde{u}} \leftarrow \vector{u}$.

	\item Vector $\vector{\widetilde{v}} \leftarrow \vector{v}$.
\end{enumerate}

We are now ready to carry out the element-wise ``product''.

The intermediate vector $\vector{\widetilde{t}} = \langle
\bDout({\sf op}), \bold{size}(\vector{\widetilde{u}}),
\{(i,t_i) : \bold{ind}(\vector{\widetilde{u}}) \cap 
\bold{ind}(\vector{\widetilde{v}})
 \neq \emptyset \} \rangle$
is created.  The value of each of its elements is computed by:
\[t_i = (\vector{\widetilde{u}}(i)
\otimes \vector{\widetilde{v}}(i)), \forall i \in 
(\bold{ind}(\vector{\widetilde{u}}) \cap 
\bold{ind}(\vector{\widetilde{v}}))\]

%-----------------------------------------------------------------------------

\subsection{{\sf eWiseMult/Intersect}: Matrix variant}

Perform element-wise (general) multiplication on the intersection of elements 
of two matrices, producing a third matrix as result.

\paragraph{\syntax}


$$
\matrix{T} ~ = ~ \matrix{A}^{[T]} \otimes \matrix{B}^{[T]}
$$

where

\begin{itemize}[leftmargin=1.1in]
    \item[$\matrix{A}^{[T]}$]    ({\sf IN}) $\in \mathbb{D}_{A}^{n\times m}$

    \item[$\matrix{B}^{[T]}$]    ({\sf IN}) $\in \mathbb{D}_{B}^{n\times m}$

    \item[$\otimes$]   ({\sf IN}) binary operator to combine values, $\mathbb{D}_{in1} \times \mathbb{D}_{in2} \rightarrow \mathbb{D}_\otimes$.  It does not need to commutative nor associative.

    \item[$\matrix{T}$]    ({\sf OUT}) $\in \mathbb{D}_\otimes^{n\times m}$
\end{itemize}


\paragraph{Description}

This variant of {\sf eWiseMult} or {\sf Intersect} computes the element-wise ``product'' of
two matrices: $\matrix{T} = \matrix{A} \otimes \matrix{B}$.  Assuming the domains of the
elements in the matrices and the domains expected by the operator are compatible
and the sizes of the matrices (after optional transposes) are the same.

From the argument matrices, the internal matrices used in 
the computation are formed ($\leftarrow$ denotes copy): \scott{Find a notation that does not refer to descriptors (descriptors is an implementation detail not math).}
\begin{enumerate}
	\item Matrix $\matrix{\widetilde{A}} \leftarrow
    {\sf desc[GrB\_INP0].GrB\_TRAN} \ ? \ \matrix{A}^T : \matrix{A}$.

	\item Matrix $\matrix{\widetilde{B}} \leftarrow
    {\sf desc[GrB\_INP1].GrB\_TRAN} \ ? \ \matrix{B}^T : \matrix{B}$.
\end{enumerate}

The intermediate matrix $\matrix{\widetilde{T}} = \langle
\bDout({\sf op}), \bold{nrows}(\matrix{\widetilde{A}}), \bold{ncols}(\matrix{\widetilde{A}}),
\{(i,j,T_{ij}) : \bold{ind}(\matrix{\widetilde{A}}) \cap 
\bold{ind}(\matrix{\widetilde{B}}) \neq \emptyset \} \rangle$
is created.  The value of each of its elements is computed by 
\[T_{ij} = (\matrix{\widetilde{A}}(i,j)
\otimes \matrix{\widetilde{B}}(i,j)), \forall (i,j) \in 
\bold{ind}(\matrix{\widetilde{A}}) \cap 
\bold{ind}(\matrix{\widetilde{B}})\]

%=============================================================================
%-----------------------------------------------------------------------------
\section{Element-wise addition or union}

{\sf eWiseAdd} returns an object whose set of indices are the ``union'' of the 
indices of the input operand. The operator is invoked when scalars from two 
operands are present (either stored in the matrix or in the optional variant: 
passed as the fill-in value).


%-----------------------------------------------------------------------------
\subsection{{\sf eWiseAdd/Union}: Vector variant}

Perform element-wise (general) addition on the elements of two vectors,
producing a third vector as result.

\paragraph{\syntax}

\begin{eqnarray*}
\matrix{T} & = & \vector{u} \oplus \vector{v} \\
\matrix{T} & = & \left\{ \vector{u}\cup u_0 \right\} \oplus \left\{ \vector{v} \cup v_0 \right\}  \mbox{~~~(providing missing values)}
\end{eqnarray*}

where

\begin{itemize}[leftmargin=1.1in]
    \item[$\vector{u}$]    ({\sf IN}) $\in \mathbb{D}_{u}^{n}$

    \item[$\vector{v}$]    ({\sf IN}) $\in \mathbb{D}_{v}^{n}$

    \item[$\oplus$]   ({\sf IN}) binary operator to combine values, $\mathbb{D}_{in1} \times \mathbb{D}_{in2} \rightarrow \mathbb{D}_\oplus$.  It does not need to commutative or associative.

    \item[$\matrix{T}$]    ({\sf OUT}) $\in \mathbb{D}_\oplus^{n}$

\end{itemize}

\paragraph{Description}

This variant of {\sf eWiseAdd} or {\sf Union} computes the element-wise ``sum'' or
``union'' of two vectors given the specified binary operator, $\oplus$: 
$\matrix{T} = \vector{u} \oplus \vector{v}$.  Assuming the domains of the
elements in the vectors and the domains expected by the operator are compatible
and the sizes of the vectors are the same.

From the argument vectors, the internal vectors used in 
the computation are formed ($\leftarrow$ denotes copy):
\begin{enumerate}
	\item Vector $\vector{\widetilde{u}} \leftarrow \vector{u}$.

	\item Vector $\vector{\widetilde{v}} \leftarrow \vector{v}$.
\end{enumerate}

We are now ready to carry out the element-wise ``sum''.

The intermediate vector $\vector{\widetilde{t}} = \langle
\bDout({\sf op}), \bold{size}(\vector{\widetilde{u}}),
\{(i,t_i) : \bold{ind}(\vector{\widetilde{u}}) \cup 
\bold{ind}(\vector{\widetilde{v}})
 \neq \emptyset \} \rangle$
is created.  The value of each of its elements is computed by:
\begin{eqnarray*}
t_i & = & 
\begin{cases}
 \vector{\widetilde{u}}(i) \oplus \vector{\widetilde{v}}(i), & \forall i \in (\bold{ind}(\vector{\widetilde{u}}) \cap \bold{ind}(\vector{\widetilde{v}}))\\
 \vector{\widetilde{u}}(i), & \forall i \in (\bold{ind}(\vector{\widetilde{u}}) - (\bold{ind}(\vector{\widetilde{u}}) \cap \bold{ind}(\vector{\widetilde{v}})))\\
  \vector{\widetilde{v}}(i), & \forall i \in (\bold{ind}(\vector{\widetilde{v}}) - (\bold{ind}(\vector{\widetilde{u}}) \cap \bold{ind}(\vector{\widetilde{v}})))
\end{cases}
\end{eqnarray*}
where the difference operator in the previous expressions refers to set difference.

In the case where $u_0$ and $v_0$ scalars are specified they are used when only
one of the vector operands has a stored value at a particular location (and the other does).  
The value of each of its elements is computed by:
\begin{eqnarray*}
t_i & = & 
\begin{cases}
 \vector{\widetilde{u}}(i) \oplus \vector{\widetilde{v}}(i), & \forall i \in (\bold{ind}(\vector{\widetilde{u}}) \cap \bold{ind}(\vector{\widetilde{v}}))\\
 \vector{\widetilde{u}}(i) {\color{red}~ \oplus ~ v_0}, & \forall i \in (\bold{ind}(\vector{\widetilde{u}}) - (\bold{ind}(\vector{\widetilde{u}}) \cap \bold{ind}(\vector{\widetilde{v}})))\\

 {\color{red} u_0 ~\oplus ~}\vector{\widetilde{v}}(i), & \forall i \in (\bold{ind}(\vector{\widetilde{v}}) - (\bold{ind}(\vector{\widetilde{u}}) \cap \bold{ind}(\vector{\widetilde{v}})))
\end{cases}
\end{eqnarray*}
where the difference operator in the previous expressions refers to set difference.


%-----------------------------------------------------------------------------

\subsection{{\sf eWiseAdd/Union}: Matrix variant}

Perform element-wise (general) addition on the elements of two matrices,
producing a third matrix as result.

\paragraph{\syntax}

\begin{eqnarray*}
\matrix{T} & = & \matrix{A}^{[T]} \oplus \matrix{B}^{[T]} \\
\matrix{T} & = & \left\{ \matrix{A}^{[T]}\cup a_0 \right\} \oplus \left\{ \matrix{B}^{[T]} \cup b_0 \right\}  \mbox{~~~(providing missing values)}
\end{eqnarray*}

where

\begin{itemize}[leftmargin=1.1in]
    \item[$\matrix{A}^{[T]}$]    ({\sf IN}) $\in \mathbb{D}_{A}^{n\times m}$

    \item[$\matrix{B}^{[T]}$]    ({\sf IN}) $\in \mathbb{D}_{B}^{n\times m}$

    \item[$\oplus$]   ({\sf IN}) binary operator to combine values, $\mathbb{D}_{in1} \times \mathbb{D}_{in2} \rightarrow \mathbb{D}_\oplus$.  It does not need to commutative or associative.

    \item[$\matrix{T}$]    ({\sf OUT}) $\in \mathbb{D}_\otimes^{n\times m}$
\end{itemize}

\paragraph{Description}

This variant of {\sf eWiseAdd} or {\sf Union} computes the element-wise ``sum'' or
``union'' of two matrices given the specified binary operator, $\oplus$: 
$\matrix{T} = {\sf A} \oplus {\sf B}$.  Assuming the domains of the
elements in the matrices and the domains expected by the operator are compatible
and the sizes of the matrices are the same.

From the argument matrices, the internal matrices used in 
the computation are formed ($\leftarrow$ denotes copy): \scott{Find a notation that does not refer to descriptors (descriptors is an implementation detail not math).}
\begin{enumerate}	\item Matrix $\matrix{\widetilde{A}} \leftarrow
    {\sf desc[GrB\_INP0].GrB\_TRAN} \ ? \ {\sf A}^T : {\sf A}$.

	\item Matrix $\matrix{\widetilde{B}} \leftarrow
    {\sf desc[GrB\_INP1].GrB\_TRAN} \ ? \ {\sf B}^T : {\sf B}$.
\end{enumerate}

The intermediate matrix $\matrix{\widetilde{T}} = \langle
\bDout({\sf op}), \bold{nrows}(\matrix{\widetilde{A}}), \bold{ncols}(\matrix{\widetilde{A}}),
\{(i,j,T_{ij}) : \bold{ind}(\matrix{\widetilde{A}}) \cup 
\bold{ind}(\matrix{\widetilde{B}}) \neq \emptyset \} \rangle$
is created.  The value of each of its elements is computed by :
\begin{eqnarray*}
T_{ij} & = & 
\begin{cases}
 (\matrix{\widetilde{A}}(i,j) \oplus \matrix{\widetilde{B}}(i,j)),& \forall (i,j) \in \bold{ind}(\matrix{\widetilde{A}}) \cap \bold{ind}(\matrix{\widetilde{B}})\\
 \matrix{\widetilde{A}}(i,j),& \forall (i,j) \in (\bold{ind}(\matrix{\widetilde{A}}) - (\bold{ind}(\matrix{\widetilde{A}}) \cap \bold{ind}(\matrix{\widetilde{B}})))\\
 \matrix{\widetilde{B}}(i.j),& \forall (i,j) \in (\bold{ind}(\matrix{\widetilde{B}}) - (\bold{ind}(\matrix{\widetilde{A}}) \cap \bold{ind}(\matrix{\widetilde{B}})))
\end{cases}
\end{eqnarray*}
where the difference operator in the previous expressions refers to set difference.

In the case where $a_0$ and $b_0$ scalars are specified they are used when only
one of the vector operands has a stored value at a particular location (and the other does).  
The value of each of its elements is computed by:
\begin{eqnarray*}
T_{ij} & = & 
\begin{cases}
 (\matrix{\widetilde{A}}(i,j) \oplus \matrix{\widetilde{B}}(i,j)), & \forall (i,j) \in \bold{ind}(\matrix{\widetilde{A}}) \cap \bold{ind}(\matrix{\widetilde{B}})\\
 \matrix{\widetilde{A}}(i,j) {\color{red} ~\oplus~ b_0}, & \forall (i,j) \in (\bold{ind}(\matrix{\widetilde{A}}) - (\bold{ind}(\matrix{\widetilde{A}}) \cap \bold{ind}(\matrix{\widetilde{B}})))\\
 {\color{red} a_0 ~\oplus ~}\matrix{\widetilde{B}}(i.j), & \forall (i,j) \in (\bold{ind}(\matrix{\widetilde{B}}) - (\bold{ind}(\matrix{\widetilde{A}}) \cap \bold{ind}(\matrix{\widetilde{B}})))
\end{cases}
\end{eqnarray*}
where the difference operator in the previous expressions refers to set difference.
