%-----------------------------------------------------------------------------
\section{{\sf mxm}: Matrix-matrix multiply}

Multiplies a matrix with another matrix on a semiring. The result is a matrix.

\paragraph{\syntax}

$$
\matrix{T} ~ = ~ \matrix{A}^{[T]} \oplus.\otimes \matrix{B}^{[T]}
$$

\paragraph{Given:}

\begin{itemize}[leftmargin=0.7in]
    \item[$\matrix{T}$]    ({\sf INOUT}) $\in \mathbb{D}_\oplus^{n\times m}$
    
    \item[$\oplus.\otimes$]   ({\sf IN}) The semiring used in the matrix
    multiplication, $\langle \mathbb{D}_\oplus, \mathbb{D}_{in1},\mathbb{D}_{in2},\oplus,\otimes,0 \rangle$.

    \item[$\matrix{A}^{[T]}$]    ({\sf IN}) $\in \mathbb{D}_A^{n\times k}$

    \item[$\matrix{B}^{[T]}$]    ({\sf IN}) $\in \mathbb{D}_B^{k\times m}$

\end{itemize}


\paragraph{Description}

{\sf mxm} computes the matrix product ${\sf T} = {\sf
A}^{[T]} \oplus . \otimes {\sf B}^{[T]}$ (where matrices {\sf A}
and {\sf B} can be optionally transposed).  

\scott{Definitions/notation of each matrix omitted, define once earlier if at all.}

\scott{Domain compatibility b/w containers and operators omitted, define once earlier.}


From the argument matrices, the internal matrices used in 
the computation are formed ($\leftarrow$ denotes copy): \scott{Find a notation that does not refer to descriptors (descriptors is an implementation detail not math).}
\begin{enumerate}

	\item Matrix $\matrix{\widetilde{A}} \leftarrow
    {\sf desc[GrB\_INP0].GrB\_TRAN} \ ? \ {\sf A}^T : {\sf A}$.

	\item Matrix $\matrix{\widetilde{B}} \leftarrow
    {\sf desc[GrB\_INP1].GrB\_TRAN} \ ? \ {\sf B}^T : {\sf B}$.
\end{enumerate}

\scott{Dimension compatibility checks omitted.}

We are now ready to carry out the matrix multiplication.
The intermediate matrix \\
$\matrix{\widetilde{T}} = \langle
\bDout({\sf op}), \bold{nrows}(\matrix{\widetilde{A}}), \bold{ncols}(\matrix{\widetilde{B}}),
\{(i,j,T_{ij}) : \bold{ind}(\matrix{\widetilde{A}}(i,:)) \cap
\bold{ind}(\matrix{\widetilde{B}}(:,j)) \neq \emptyset \} \rangle$
is created.  The value of each of its elements is computed by 
\[T_{ij} = \bigoplus_{k \in \bold{ind}(\matrix{\widetilde{A}}(i,:)) \cap
\bold{ind}(\matrix{\widetilde{B}}(:,j))} (\matrix{\widetilde{A}}(i,k)
\otimes \matrix{\widetilde{B}}(k,j)),\] where $\oplus$ and $\otimes$
are the additive and multiplicative operators of semiring {\sf op},
respectively.

\scott{Is the following describe possibilities in the realm of math?}
\begin{itemize}
\item if $\oplus$ is commutative and associative no order is imposed on the elements of the intersection set.
\item if $\oplus$ is not commutative then the order must be from lowest to highest index $k$. Arbitrary grouping is allowed as long as relative order is maintained.
\item if $\oplus$ is neither commutative nor associative, strictly sequential and ordered operation is required.
\end{itemize}

%-----------------------------------------------------------------------------

\section{{\sf vxm}: Vector-matrix multiply}

Multiplies a (row) vector with a matrix on an semiring. The result is a vector.

\paragraph{\syntax}

$$
\vector{t}^T ~ = ~ \vector{u}^{T} \oplus.\otimes \matrix{A}^{[T]}
$$

\paragraph{Parameters}

\begin{itemize}[leftmargin=1.1in]
    \item[$\vector{t}$]    ({\sf INOUT}) $\in \mathbb{D}_\oplus^{m}$

    \item[$\oplus.\otimes$]   ({\sf IN}) The semiring used in the matrix
    multiplication, $\langle \mathbb{D}_\oplus, \mathbb{D}_{in1},\mathbb{D}_{in2},\oplus,\otimes,0 \rangle$.

    \item[$\vector{u}$]    ({\sf INOUT}) $\in \mathbb{D}_{u}^{n}$

    \item[$\matrix{A}^{[T]}$]    ({\sf IN}) $\in \mathbb{D}_A^{n\times m}$

\end{itemize}

\paragraph{Description}

{\sf vxm} computes the vector-matrix product ${\sf t}^T = {\sf
u}^T \oplus . \otimes {\sf A}^{[T]}$ (where matrix {\sf A}
 can be optionally transposed).

\scott{Definitions/notation of each matrix and vector omitted, define once earlier if at all.}

\scott{Domain compatibility b/w containers and operators omitted, define once earlier.}


From the argument vectors and matrices, the internal matrices and vectors used in 
the computation are formed ($\leftarrow$ denotes copy): \scott{Find a notation that does not refer to descriptors (descriptors is an implementation detail not math).}
\begin{enumerate}
	\item Vector $\vector{\widetilde{u}} \leftarrow {\sf u}$.

	\item Matrix $\matrix{\widetilde{A}} \leftarrow {\sf desc[GrB\_INP1].GrB\_TRAN} \ ? \ {\sf A}^T : {\sf A}$.
\end{enumerate}

\scott{Dimension compatibility checks omitted.}


We are now ready to carry out the vector-matrix multiplication.

The intermediate vector $\vector{\widetilde{t}} = \langle
\bDout({\sf op}), \bold{ncols}(\matrix{\widetilde{A}}),
\{(j,t_j) : \bold{ind}(\vector{\widetilde{u}}) \cap
\bold{ind}(\matrix{\widetilde{A}}(:,j)) \neq \emptyset \} \rangle$
is created.  The value of each of its elements is computed by 
\[t_j = \bigoplus_{k \in \bold{ind}(\vector{\widetilde{u}}) \cap
\bold{ind}(\matrix{\widetilde{A}}(:,j))} (\vector{\widetilde{u}}(k)
\otimes \matrix{\widetilde{A}}(k,j)),\] where $\oplus$ and $\otimes$
are the additive and multiplicative operators of semiring {\sf op},
respectively.

\scott{Same comments about associativity and commutativity.}

%-----------------------------------------------------------------------------

\section{{\sf mxv}: Matrix-vector multiply}

Multiplies a matrix by a vector on a semiring. The result is a vector.

\paragraph{\syntax}

$$
\vector{t}~ = ~ \matrix{A}^{[T]} \oplus.\otimes \vector{u}
$$

\paragraph{Parameters}

\begin{itemize}[leftmargin=1.1in]
    \item[$\vector{t}$]    ({\sf INOUT}) $\in \mathbb{D}_\oplus^{n}$

    \item[$\oplus.\otimes$]   ({\sf IN}) The semiring used in the matrix
    multiplication, $\langle \mathbb{D}_\oplus, \mathbb{D}_{in1},\mathbb{D}_{in2},\oplus,\otimes,0 \rangle$.

    \item[$\matrix{A}^{[T]}$]    ({\sf IN}) $\in \mathbb{D}_A^{n\times m}$

    \item[$\vector{u}$]    ({\sf INOUT}) $\in \mathbb{D}_{u}^{m}$

\end{itemize}

\paragraph{Description}

{\sf GrB\_mxv} computes the matrix-vector product ${\sf w} = {\sf A}
\oplus . \otimes {\sf u}$ (where matrix {\sf A}
 can be optionally transposed).

\scott{Definitions/notation of each matrix and vector omitted, define once earlier if at all.}

\scott{Domain compatibility b/w containers and operators omitted, define once earlier.}

From the argument vectors and matrices, the internal matrices and mask used in 
the computation are formed ($\leftarrow$ denotes copy):  \scott{Find a notation that does not refer to descriptors (descriptors is an implementation detail not math).}
\begin{enumerate}
	\item Matrix $\matrix{\widetilde{A}} \leftarrow {\sf desc[GrB\_INP0].GrB\_TRAN} \ ? \ {\sf A}^T : {\sf A}$.

	\item Vector $\vector{\widetilde{u}} \leftarrow {\sf u}$.
\end{enumerate}

\scott{Dimension compatibility checks omitted.}

We are now ready to carry out the matrix-vector multiplication.

The intermediate vector $\vector{\widetilde{t}} = \langle
\bDout({\sf op}), \bold{nrows}(\matrix{\widetilde{A}}),
\{(i,t_i) : \bold{ind}(\matrix{\widetilde{A}}(i,:)) \cap 
\bold{ind}(\vector{\widetilde{u}}) \neq \emptyset \} \rangle$
is created.  The value of each of its elements is computed by 
\[t_i = \bigoplus_{k \in \bold{ind}(\matrix{\widetilde{A}}(i,:)) \cap
\bold{ind}(\vector{\widetilde{u}})} (\matrix{\widetilde{A}}(i,k)
\otimes \vector{\widetilde{u}}(k)),\] where $\oplus$ and $\otimes$
are the additive and multiplicative operators of semiring {\sf op},
respectively.

\scott{Same comments about associativity and commutativity.}
