\chapter{GraphBLAS Operations}
\label{Sec:Operations}

The GraphBLAS operations are defined in the GraphBLAS math specification and summarized in 
Table~\ref{Tab:GraphBLASOps}.   In addition to methods that implement these
fundamental GraphBLAS operations, we support a number of variants that have been 
found to be especially useful in algorithm development.
A flowchart of the overall behavior of a GraphBLAS operation is shown 
in Figure~\ref{Fig:mxmFlowchart}.

\subparagraph{Associativity} 
We would like to emphasize that no GraphBLAS operation will imply a predefined order over any associative operators. Implementations of the GraphBLAS are encouraged to exploit associativity to optimize performance of any GraphBLAS operation. This holds even if the definition of the operation implies a fixed order for the associative operations.


\begin{table}[p]
\hrule
\begin{center}
\caption[A mathematical notation for the fundamental GraphBLAS operations 
supported in this specification.]{A mathematical notation for the fundamental GraphBLAS operations 
supported in this specification.  Input matrices $\matrix{A}$ and $\matrix{B}$ 
may be optionally transposed (not shown). Use of an optional accumulate with 
existing values in the output object is indicated with $\odot$.  Use of optional write 
masks and replace flags are indicated as $\matrix{C}\langle\matrix{M},r\rangle$ 
when applied to the output matrix, $\matrix{C}$.  The mask controls which values 
resulting from the operation on the right-hand side are written into the output 
object (complement and structure flags are not shown).  The ``replace'' 
option, indicated by specifying the $r$ flag, means that all values in the 
output object are removed prior to assignment. If ``replace'' is not specified, 
only the values/locations computed on the right-hand side and allowed by the 
mask will be written to the output (``merge'' mode).}
\label{Tab:GraphBLASOps}
~\\
\newcommand{\odotsp}{\hspace{-0.2cm}\odot\hspace{-0.18cm}}
\begin{tabular}{l|rcrcl}
{\sf Operation Name} & \multicolumn{5}{c}{Mathematical Notation}  \\
\hline
{\sf mxm}          & $\matrix{C}\langle\matrix{M},r\rangle$ & $=$ & $\matrix{C}$ & $\odotsp$ & $\matrix{A} \oplus.\otimes \matrix{B}$  \\
{\sf mxv}          & $\vector{w}\langle\vector{m},r\rangle$ & $=$ & $\vector{w}$ & $\odotsp$ & $\matrix{A} \oplus.\otimes \vector{u}$  \\
{\sf vxm}          & $\vector{w}^T\langle\vector{m}^T,r\rangle$ & $=$ & \hspace{-0.18cm}$\vector{w}^T$ & $\odotsp$ & $\vector{u}^T \oplus.\otimes \matrix{A}$  \\
{\sf eWiseMult}    & $\matrix{C}\langle\matrix{M},r\rangle$ & $=$ & $\matrix{C}$ & $\odotsp$ & $\matrix{A} \otimes \matrix{B}$  \\
                   & $\vector{w}\langle\matrix{m},r\rangle$ & $=$ & $\vector{w}$ & $\odotsp$ & $\vector{u} \otimes \vector{v}$  \\
{\sf eWiseAdd}     & $\matrix{C}\langle\matrix{M},r\rangle$ & $=$ & $\matrix{C}$ & $\odotsp$ & $\matrix{A} \oplus  \matrix{B}$  \\
                   & $\vector{w}\langle\matrix{m},r\rangle$ & $=$ & $\vector{w}$ & $\odotsp$ & $\vector{u} \oplus \vector{v}$  \\
{\sf extract}      & $\matrix{C}\langle\matrix{M},r\rangle$ & $=$ & $\matrix{C}$ & $\odotsp$ & $\matrix{A}(\grbarray{i},\grbarray{j})$ \\
                   & $\vector{w}\langle\matrix{m},r\rangle$ & $=$ & $\vector{w}$ & $\odotsp$ & $\vector{u}(\grbarray{i})$ \\
%{\sf extract} (column) & $\matrix{w}\langle\vector{m},r\rangle$ & $=$ & $\matrix{w}$ & $\odotsp$ & $\matrix{A}(\grbarray{i}, j)$ \\
{\sf assign}       & $\matrix{C}\langle\matrix{M},r\rangle(\grbarray{i},\grbarray{j})$ & $=$ & $\matrix{C}(\grbarray{i},\grbarray{j})$ & $\odotsp$ & $\matrix{A}$ \\
                   & $\vector{w}\langle\vector{m},r\rangle(\grbarray{i})$ & $=$ & $\vector{w}(\grbarray{i})$ & $\odotsp$ & $\matrix{u}$ \\
{\sf reduce} (row) & $\vector{w}\langle\vector{m},r\rangle$ & $=$ & $\vector{w}$ & $\odotsp$ & $\left[\oplus_j\matrix{A}(:,j)\right]$  \\
{\sf reduce} (scalar) & $s$ & $=$ & $s$ & $\odotsp$ & $\left[\oplus_{i,j}\matrix{A}(i,j) \right]$  \\
                      & $s$ & $=$ & $s$ & $\odotsp$ & $\left[\oplus_i\matrix{u}(i) \right]$  \\
{\sf apply}        & $\matrix{C}\langle\matrix{M},r\rangle$ & $=$ & $\matrix{C}$ & $\odotsp$ & $f_u(\matrix{A})$ \\
                   & $\vector{w}\langle\matrix{m},r\rangle$ & $=$ & $\vector{w}$ & $\odotsp$ & $f_u(\vector{u} )$  \\
\hline
{\sf apply(indexop)}     & $\matrix{C}\langle\matrix{M},r\rangle$ & $=$ & $\matrix{C}$ & $\odotsp$ & $f_{i}(\matrix{A},\mathbf{ind}(\matrix{A}),s)$ \\
                   & $\vector{w}\langle\matrix{m},r\rangle$ & $=$ & $\vector{w}$ & $\odotsp$ & $f_{i}(\vector{u},\mathbf{ind}(\vector{u}),s)$  \\
{\sf select  }     & $\matrix{C}\langle\matrix{M},r\rangle$ & $=$ & $\matrix{C}$ & $\odotsp$ & $\matrix{A}\langle f_{i}(\matrix{A},\mathbf{ind}(\matrix{A}),s)\rangle$ \\
                   & $\vector{w}\langle\matrix{m},r\rangle$ & $=$ & $\vector{w}$ & $\odotsp$ & $\vector{u}\langle f_{i}(\vector{u},\mathbf{ind}(\vector{u}),s)\rangle$  \\
\hline
{\sf transpose}    & $\matrix{C}\langle\matrix{M},r\rangle$ & $=$ & $\matrix{C}$ & $\odotsp$ & $\matrix{A}^T$ \\
{\sf kronecker}          & $\matrix{C}\langle\matrix{M},r\rangle$ & $=$ & $\matrix{C}$ & $\odotsp$ & $\matrix{A}  \kron \matrix{B}$  \\
%& & & \\
%& \multicolumn{3}{c}{Input/Output Operations} \\
%{\sf Matrix\_build}  & $\matrix{C}$ & $=$ & $\mathbb{S}^{m\times n}(\grbarray{i},\grbarray{j},\grbarray{v},\oplus_{dup})$ \\
%{\sf Vector\_build}  & $\vector{w}$ & $=$ & $\mathbb{S}^{n}(\grbarray{i},\grbarray{v},\oplus_{dup})$ \\
%{\sf Matrix\_extractTuples} & $(\grbarray{i},\grbarray{j},\grbarray{v})$ & $=$ & $\matrix{A}$ \\
%{\sf Vector\_extractTuples} & $(\grbarray{i},\grbarray{v})$ & $=$ & $\matrix{u}$ \\
\end{tabular}
\end{center}
\hrule
\end{table}

% Another take on the graphblas math notation
\input{graphblas_notation_table}


\begin{figure}[p]
    \hrule
    \begin{center}
        \includegraphics[width=5.5in]{mxm_operation_flowchart_1_3d.pdf}
    \end{center}
    \caption[Flowchart for the GraphBLAS operations.]{Flowchart for the GraphBLAS operations. Although shown specifically for
	the {\sf mxm} operation, many elements are common to all operations: such as the 
	``{\sf ACCUM}'' and ``{\sf MASK and REPLACE}'' blocks.  The triple arrows 
    ($\Rrightarrow$) denote where ``as if copy'' takes place (including both 
    collections and descriptor settings).  The bold, dotted arrows indicate
    where casting may occur between different domains.}
    \label{Fig:mxmFlowchart}
    \hrule
\end{figure}

\subparagraph{Domains \scott{and Casting?}:}
A GraphBLAS operation is only valid when the domains of the GraphBLAS objects are
mathematically consistent.  A programming language may define implicit casts 
between data types.  For example, {\sf float}s, {\sf double}s, and {\sf int}s can be 
freely mixed according to the rules defined for implicit casts in the C language.  It is the 
responsibility of the user to assure that these casts are appropriate for the language
in question.  For example, a cast to {\sf int} implies truncation of a floating 
point type.  Depending on the operation, this truncation error could lead to
erroneous results.  Furthermore, casting a wider type onto a narrower type can lead 
to overflow errors.  The specification of the GraphBLAS operations do not attempt to 
protect a user from these sorts of errors.

\subparagraph{Dimensions and Transposes:}
GraphBLAS operations also make assumptions about the numbers of dimensions and 
the sizes of vectors and matrices in an operation.   In this document we assume the
operands are \emph{shape compatible} for the mathematical definition of the operation 
being specified.  For example, when multiplying two matrices, 
$\matrix{C} = \matrix{A} \times \matrix{B}$, the number 
of rows of $\matrix{C}$ must equal the number of rows of $\matrix{A}$, the number of 
columns of $\matrix{A}$ must match the number of rows of $\matrix{B}$, and the number 
of columns of $\matrix{C}$ must match the number of columns of $\matrix{B}$.   
For most of the GraphBLAS operations involving matrices, optional transposition of
matrices can be specified. When matrices are optionally transposed \emph{shape compatibility}
is assumed to hold for the transposed matrix.

All GraphBLAS operations performed ``as if'' they are carried out in three sequential 
steps outlined as follows:

\begin{enumerate}[leftmargin=1.1in]
\item[\bf Base] The specific computation of a particular operation is carried out.  
An example is $\matrix{A}\oplus.\otimes\matrix{B}$ of {\sf mxm} or 
$\left[\oplus_{i,j}\matrix{A}(i,j) \right]$ of {\sf reduce}.  In this specification, 
the result of this step is stored in an intermediate matrix ($\tilde{\matrix{T}}$), 
vector ($\tilde{\vector{t}}$), or scalar ($\tilde{t}$).

\item[\bf Accumulation] All GraphBLAS operations support an optional accumulation step 
adding values in the existing output container (scalar, vector or matrix) with the 
results of the Base step.  In this specification, the result of this step is stored in 
an intermediate matrix ($\tilde{\matrix{Z}}$), vector ($\tilde{\vector{z}}$), or scalar ($\tilde{z}$).

\item[\bf Masking] For matrix and vector operations, the final result is written into 
the output container, possibly under control of a write mask -- either $\matrix{M}$ for 
matrices or $\vector{m}$ for vectors.  Scalar operations do not carry out this step.
\end{enumerate}

The Accumulation and Masking steps are common to most of the operations and are presented separately first.


%============================================================
%============================================================
\section{Optional Accumulation}

The mathematics...

\begin{itemize}
    \item[$\odot$] ({\sf IN}) An optional binary operator used for accumulating
    entries into existing $\matrix{C}$ entries, $\langle \bDout(\odot),\bDin1(\odot),\bDin2(\odot), \odot \rangle$.
\end{itemize}

\paragraph{Matrix accumulation}

\input{ops_accum_z_matrix}


\paragraph{Vector accumulation}

\input{ops_accum_z_vector}


%============================================================
%============================================================
\section{Optional Masking}

The mathematics...

\begin{itemize}
    \item[$\tilde{\matrix{M}}$] ({\sf IN}) $\in \mathbb{B}^{n\times m}$

	\item[{\sf Mask}] $\langle \bold{D}({\sf Mask}),\bold{nrows}({\sf Mask}),\bold{ncols}({\sf Mask}),\bold{L}({\sf Mask}) = \{(i,j,M_{ij}) \} \rangle$ (optional)
\end{itemize}

\paragraph{Masks: Structure-only, Complement, and Replace}

When a GraphBLAS operation supports the use of an optional mask, that mask is
specified through a GraphBLAS vector (for one-dimensional masks) or
a GraphBLAS matrix (for two-dimensional masks).  When a mask is used and the 
{\tt GrB\_STRUCTURE} descriptor value is not set, it is applied to the result 
from the operation wherever the stored values in the mask evaluate to true.  If
the {\tt GrB\_STRUCTURE} descriptor is set, the mask is applied to the result
from the operation wherever the mask as a stored value (regardless of that value).
Wherever the mask is applied, the result from the operation is either assigned 
to the provided output matrix/vector or, if a binary accumulation operation is 
provided, the result is accumulated into the corresponding elements of the provided 
output matrix/vector.

Given a GraphBLAS vector $\vector{v} = \langle D,N, \{ (i,v_i) \} \rangle$, a
one-dimensional mask is derived for use in the operation as follows:
\[
\vector{m} = 
\begin{cases}
\langle N, \{ \mathbf{ind}(\vector{v}) \} \rangle, & \mbox{if {\tt GrB\_STRUCTURE} is specified,} \\
\langle N, \{ i : \mbox{\tt (bool)}v_i = \true \} \rangle, & \mbox{otherwise}
\end{cases}
\]
where {\tt (bool)}$v_i$ denotes casting the value $v_i$ to a Boolean value (\true\ or \false).
Likewise, given a GraphBLAS matrix $\matrix{A} = \langle D, M, N, \{ (i,j,A_{ij}) \} \rangle$,
a two-dimensional mask is derived for use in the operation as follows:
\[
\matrix{M} = 
\begin{cases}
\langle M,N, \{ \mathbf{ind}(\matrix{A}) \} \rangle, & \mbox{if {\tt GrB\_STRUCTURE} is specified,} \\
\langle M,N, \{ (i,j) : \mbox{\tt (bool)}A_{ij} = \true \} \rangle, & \mbox{otherwise}
\end{cases}
\]
where {\tt (bool)}$A_{ij}$ denotes casting the value $A_{ij}$ to a Boolean value. 
(\true\ or \false)

In both the one- and two-dimensional cases, the mask may also have a subsequent 
complement operation applied ($Section$~\ref{Sec:Masks}) as specified in the 
descriptor, before a final mask is generated for use in the operation.

When the descriptor of an operation with a mask has specified that 
the {\sf GrB\_REPLACE} value is to be applied to the output ({\sf GrB\_OUTP}),
then anywhere the mask is not {\sf true}, the corresponding location in
the output is cleared.

$$
\matrix{C} \langle \matrix{M} \rangle \odot\!\!= \matrix{A} \oplus.\otimes \matrix{B}
$$

$$
\matrix{C} \langle \neg{\matrix{M}} \rangle \odot\!\!= \matrix{A} \oplus.\otimes \matrix{B}
$$

$$
\matrix{C} \langle s(\matrix{M}) \rangle \odot\!\!= \matrix{A} \oplus.\otimes \matrix{B}
$$

$$
\matrix{C} \langle s(\neg{\matrix{M}} \rangle \odot\!\!= \matrix{A} \oplus.\otimes \matrix{B}
$$


From the argument matrices, the internal mask used in 
the computation are formed ($\leftarrow$ denotes copy):
\begin{enumerate}
	\item Matrix $\matrix{\widetilde{C}} \leftarrow {\sf C}$.

	\item Two-dimensional mask, $\matrix{\widetilde{M}}$, is computed from
    argument {\sf Mask} as follows:
	\begin{enumerate}
		\item If ${\sf Mask} = {\sf GrB\_NULL}$, then $\matrix{\widetilde{M}} = 
        \langle \bold{nrows}({\sf C}), \bold{ncols}({\sf C}), \{(i,j), 
        \forall i,j : 0 \leq i <  \bold{nrows}({\sf C}), 0 \leq j < 
        \bold{ncols}({\sf C}) \} \rangle$.

		\item If {\sf Mask} $\ne$ {\sf GrB\_NULL},
        \begin{enumerate}
            \item If ${\sf desc[GrB\_MASK].GrB\_STRUCTURE}$ is set, then 
            $\matrix{\widetilde{M}} = \langle \bold{nrows}({\sf Mask}), 
            \bold{ncols}({\sf Mask}), \{(i,j) : (i,j) \in \bold{ind}({\sf Mask}) \} \rangle$,
            \item Otherwise, $\matrix{\widetilde{M}} = \langle \bold{nrows}({\sf Mask}), 
            \bold{ncols}({\sf Mask}), \\ \{(i,j) : (i,j) \in \bold{ind}({\sf Mask}) \wedge 
            ({\sf bool}){\sf Mask}(i,j) = \true\} \rangle$.
        \end{enumerate}

		\item	If ${\sf desc[GrB\_MASK].GrB\_COMP}$ is set, then 
        $\matrix{\widetilde{M}} \leftarrow \neg \matrix{\widetilde{M}}$.
	\end{enumerate}
\end{enumerate}

\paragraph{Matrix masking}

\input{ops_mask_replace_matrix}


\paragraph{Vector masking}

\input{ops_mask_replace_vector}
