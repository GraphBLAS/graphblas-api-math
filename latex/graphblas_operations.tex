\section{GraphBLAS operations}
\label{Sec:Operations}

The GraphBLAS operations are defined in the GraphBLAS math specification and summarized in 
Table~\ref{Tab:GraphBLASOps}.   In addition to methods that implement these
fundamental GraphBLAS operations, we support a number of variants that have been 
found to be especially useful in algorithm development.
A flowchart of the overall behavior of a GraphBLAS operation is shown 
in Figure~\ref{Fig:mxmFlowchart}.

\begin{table}[p]
\hrule
\begin{center}
\caption[A mathematical notation for the fundamental GraphBLAS operations 
supported in this specification.]{A mathematical notation for the fundamental GraphBLAS operations 
supported in this specification.  Input matrices $\matrix{A}$ and $\matrix{B}$ 
may be optionally transposed (not shown). Use of an optional accumulate with 
existing values in the output object is indicated with $\odot$.  Use of optional write 
masks and replace flags are indicated as $\matrix{C}\langle\matrix{M},r\rangle$ 
when applied to the output matrix, $\matrix{C}$.  The mask controls which values 
resulting from the operation on the right-hand side are written into the output 
object (complement and structure flags are not shown).  The ``replace'' 
option, indicated by specifying the $r$ flag, means that all values in the 
output object are removed prior to assignment. If ``replace'' is not specified, 
only the values/locations computed on the right-hand side and allowed by the 
mask will be written to the output (``merge'' mode).}
\label{Tab:GraphBLASOps}
~\\
\newcommand{\odotsp}{\hspace{-0.2cm}\odot\hspace{-0.18cm}}
\begin{tabular}{l|rcrcl}
{\sf Operation Name} & \multicolumn{5}{c}{Mathematical Notation}  \\
\hline
{\sf mxm}          & $\matrix{C}\langle\matrix{M},r\rangle$ & $=$ & $\matrix{C}$ & $\odotsp$ & $\matrix{A} \oplus.\otimes \matrix{B}$  \\
{\sf mxv}          & $\vector{w}\langle\vector{m},r\rangle$ & $=$ & $\vector{w}$ & $\odotsp$ & $\matrix{A} \oplus.\otimes \vector{u}$  \\
{\sf vxm}          & $\vector{w}^T\langle\vector{m}^T,r\rangle$ & $=$ & \hspace{-0.18cm}$\vector{w}^T$ & $\odotsp$ & $\vector{u}^T \oplus.\otimes \matrix{A}$  \\
{\sf eWiseMult}    & $\matrix{C}\langle\matrix{M},r\rangle$ & $=$ & $\matrix{C}$ & $\odotsp$ & $\matrix{A} \otimes \matrix{B}$  \\
                   & $\vector{w}\langle\matrix{m},r\rangle$ & $=$ & $\vector{w}$ & $\odotsp$ & $\vector{u} \otimes \vector{v}$  \\
{\sf eWiseAdd}     & $\matrix{C}\langle\matrix{M},r\rangle$ & $=$ & $\matrix{C}$ & $\odotsp$ & $\matrix{A} \oplus  \matrix{B}$  \\
                   & $\vector{w}\langle\matrix{m},r\rangle$ & $=$ & $\vector{w}$ & $\odotsp$ & $\vector{u} \oplus \vector{v}$  \\
{\sf extract}      & $\matrix{C}\langle\matrix{M},r\rangle$ & $=$ & $\matrix{C}$ & $\odotsp$ & $\matrix{A}(\grbarray{i},\grbarray{j})$ \\
                   & $\vector{w}\langle\matrix{m},r\rangle$ & $=$ & $\vector{w}$ & $\odotsp$ & $\vector{u}(\grbarray{i})$ \\
%{\sf extract} (column) & $\matrix{w}\langle\vector{m},r\rangle$ & $=$ & $\matrix{w}$ & $\odotsp$ & $\matrix{A}(\grbarray{i}, j)$ \\
{\sf assign}       & $\matrix{C}\langle\matrix{M},r\rangle(\grbarray{i},\grbarray{j})$ & $=$ & $\matrix{C}(\grbarray{i},\grbarray{j})$ & $\odotsp$ & $\matrix{A}$ \\
                   & $\vector{w}\langle\vector{m},r\rangle(\grbarray{i})$ & $=$ & $\vector{w}(\grbarray{i})$ & $\odotsp$ & $\matrix{u}$ \\
{\sf reduce} (row) & $\vector{w}\langle\vector{m},r\rangle$ & $=$ & $\vector{w}$ & $\odotsp$ & $\left[\oplus_j\matrix{A}(:,j)\right]$  \\
{\sf reduce} (scalar) & $s$ & $=$ & $s$ & $\odotsp$ & $\left[\oplus_{i,j}\matrix{A}(i,j) \right]$  \\
                      & $s$ & $=$ & $s$ & $\odotsp$ & $\left[\oplus_i\matrix{u}(i) \right]$  \\
{\sf apply}        & $\matrix{C}\langle\matrix{M},r\rangle$ & $=$ & $\matrix{C}$ & $\odotsp$ & $f_u(\matrix{A})$ \\
                   & $\vector{w}\langle\matrix{m},r\rangle$ & $=$ & $\vector{w}$ & $\odotsp$ & $f_u(\vector{u} )$  \\
\hline
{\sf apply(indexop)}     & $\matrix{C}\langle\matrix{M},r\rangle$ & $=$ & $\matrix{C}$ & $\odotsp$ & $f_{i}(\matrix{A},\mathbf{ind}(\matrix{A}),s)$ \\
                   & $\vector{w}\langle\matrix{m},r\rangle$ & $=$ & $\vector{w}$ & $\odotsp$ & $f_{i}(\vector{u},\mathbf{ind}(\vector{u}),s)$  \\
{\sf select  }     & $\matrix{C}\langle\matrix{M},r\rangle$ & $=$ & $\matrix{C}$ & $\odotsp$ & $\matrix{A}\langle f_{i}(\matrix{A},\mathbf{ind}(\matrix{A}),s)\rangle$ \\
                   & $\vector{w}\langle\matrix{m},r\rangle$ & $=$ & $\vector{w}$ & $\odotsp$ & $\vector{u}\langle f_{i}(\vector{u},\mathbf{ind}(\vector{u}),s)\rangle$  \\
\hline
{\sf transpose}    & $\matrix{C}\langle\matrix{M},r\rangle$ & $=$ & $\matrix{C}$ & $\odotsp$ & $\matrix{A}^T$ \\
{\sf kronecker}          & $\matrix{C}\langle\matrix{M},r\rangle$ & $=$ & $\matrix{C}$ & $\odotsp$ & $\matrix{A}  \kron \matrix{B}$  \\
%& & & \\
%& \multicolumn{3}{c}{Input/Output Operations} \\
%{\sf Matrix\_build}  & $\matrix{C}$ & $=$ & $\mathbb{S}^{m\times n}(\grbarray{i},\grbarray{j},\grbarray{v},\oplus_{dup})$ \\
%{\sf Vector\_build}  & $\vector{w}$ & $=$ & $\mathbb{S}^{n}(\grbarray{i},\grbarray{v},\oplus_{dup})$ \\
%{\sf Matrix\_extractTuples} & $(\grbarray{i},\grbarray{j},\grbarray{v})$ & $=$ & $\matrix{A}$ \\
%{\sf Vector\_extractTuples} & $(\grbarray{i},\grbarray{v})$ & $=$ & $\matrix{u}$ \\
\end{tabular}
\end{center}
\hrule
\end{table}

\begin{figure}[p]
    \hrule
    \begin{center}
        \includegraphics[width=5.5in]{mxm_operation_flowchart_1_3d.pdf}
    \end{center}
    \caption[Flowchart for the GraphBLAS operations.]{Flowchart for the GraphBLAS operations. Although shown specifically for
	the {\sf mxm} operation, many elements are common to all operations: such as the 
	``{\sf ACCUM}'' and ``{\sf MASK and REPLACE}'' blocks.  The triple arrows 
    ($\Rrightarrow$) denote where ``as if copy'' takes place (including both 
    collections and descriptor settings).  The bold, dotted arrows indicate
    where casting may occur between different domains.}
    \label{Fig:mxmFlowchart}
    \hrule
\end{figure}

\paragraph{Domains and Casting}

A GraphBLAS operation is only valid when the domains of the GraphBLAS objects are
mathematically consistent.  The C programming language defines implicit casts 
between built-in data types.  For example, {\sf float}s, {\sf double}s, and {\sf int}s can be 
freely mixed according to the rules defined for implicit casts.  It is the 
responsibility of the user to assure that these casts are appropriate for the 
algorithm in question.  For example, a cast to {\sf int} implies truncation of a floating 
point type.  Depending on the operation, this truncation error could lead to
erroneous results.  Furthermore, casting a wider type onto a narrower type can lead 
to overflow errors.  The GraphBLAS operations do not attempt to protect a user from 
these sorts of errors.

When user-define types are involved, however, GraphBLAS requires strict equivalence
between types and no casting is supported.  If GraphBLAS detects these mismatches,
it will return a domain mismatch error.

\paragraph{Dimensions and Transposes}

GraphBLAS operations also make assumptions about the numbers of dimensions and 
the sizes of vectors and matrices in an operation.   An operation will test these 
sizes and report an error if they are not \emph{shape compatible}.  For example, when multiplying 
two matrices, $\matrix{C} = \matrix{A} \times \matrix{B}$, the number of rows of 
$\matrix{C}$ must equal the number of rows of $\matrix{A}$, the number of columns 
of $\matrix{A}$ must match the number of rows of $\matrix{B}$, and the number of 
columns of $\matrix{C}$ must match the number of columns of $\matrix{B}$.  This 
is the behavior expected given the mathematical definition of the operations.   

For most of the GraphBLAS operations involving matrices, an optional descriptor 
can modify the matrix associated with an input GraphBLAS matrix object.  For 
example, if an input matrix is an argument to a GraphBLAS operation and the 
associated descriptor indicates the transpose option, then the operation occurs 
as if on the transposed matrix.  In this case, the relationships between the 
sizes in each dimension shift in the mathematically expected way. 


\paragraph{Compliance}

We follow a \emph{prescriptive} approach to the definition of the semantics
of GraphBLAS operations. That is, for each operation we give a recipe for
producing its outcome.
Any implementation that produces the same outcome,
and follows the GraphBLAS execution model (Section~\ref{Sec:ExecutionModel}) and
error model (Section~\ref{Sec:ErrorModel}) is a conforming implementation.
